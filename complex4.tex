\documentclass[11pt]{article}




\usepackage{fancyhdr}
\usepackage{amsthm}
\usepackage{amssymb}
\usepackage{amsmath}
\usepackage{setspace}
\pagestyle{fancyplain}



\newtheorem{theorem}{Theorem}[section]
\newtheorem*{theorem*}{Theorem}
\newtheorem{lemma}[theorem]{Lemma}
\newtheorem{proposition}[theorem]{Proposition}
\newtheorem{corollary}[theorem]{Corollary}
\newtheorem{definition}{Definition}
\begin{document}

\lhead{Frederick Robinson}
\rhead{Math 410-3: Complex Analysis}



\title{Homework 4}
\author{Frederick Robinson}
\date{27 April 2011}
\maketitle


\section*{Chapter 4}

\section{Problem 5}
\subsection{Question}
Let $P=0$. classify each of the following as having a removable singularity , a pole, or an essential singularity at $P$.
\begin{enumerate}
\item \[ \frac{1}{z}\]
\item \[\sin\frac{1}{z}\]
\item \[\frac{1}{z^3} - \cos z\]
\item \[z \cdot e^{1/z} \cdot e^{-1/z^2}\]
\item \[ \frac{\sin{z}}{z}\]
\item \[\frac{\cos z}{z}\]
\item \[ \frac{\sum_{k=2}^\infty 2^kz^k}{z^3} \]
\end{enumerate}
\subsection{Answer}
\begin{enumerate}
\item Pole. $\lim_{z \to 0} f (z)= \infty$.
\item Essential Singularity. $\lim_{z \to 0}f(z)$ does not exist.
\item Pole. $\lim_{z \to 0}f(z) = \infty$.
\item Essential Singularity. $\lim_{z \to 0}f(z)$ does not exist.
\item Removable Singularity. $\lim_{z \to 0}f(z) = 1$.
\item Pole. $\lim_{z \to 0}f(z) = \infty$.
\item Pole. $f(z) = \sum_{k=2}^\infty 2^k z^{k-3} $ which is a pole by our analysis of Laurent series.
\end{enumerate}

\section{Problem 6}
\subsection{Question}
Prove that 
\[\sum_{j=1}^\infty 2^{-(2^j)}\cdot z^{-j}\]
converges for $z \neq 0$ and defines a function which has an essential singularity at $P=0$.
\subsection{Answer}
\begin{proof}
Let $|z|=r >0$. 
\begin{align*}\sum_{j=1}^\infty |2^{-(2^j)}\cdot z^{-j} |&= \sum_{j=1}^\infty 2^{-(2^j)}\cdot r^{-j}
 \end{align*}
 Now, compute
 \[\limsup_{n \to \infty} \frac{2^{-(2^{n+1})}r ^{-(n+1)}}{2^{-(2^n)} r^{-n}} = \limsup_{n \to \infty} 2^{-(2^n)}r^{-1}=0\]
\end{proof}
Hence, the series is convergent for $|z|>0$. 

At $z=0$ there are infinitely many terms of the Laurent series with nonzero coefficient, and negative exponent. Thus, there is an essential singularity here.

\section{Problem 8}
\subsection{Question}
Let $U = D(P,r) \setminus \{ P\}$. Prove the following two refined versions of Riemann's Theorem
\begin{enumerate}
\item If $f$ is holomorphic on $U$ and $\lim_{z \to P}(z-P) \cdot f(z) =0$, then $f$ continues holomorphically across $P$ (to all of $U$).

\item If $f$ is holomorphic on $U$ and if 
\[\int_U |f(z)|^2 dx dy < \infty,\]
then $f$ extends holomorphically across $P$. 
\end{enumerate}
[\emph{Hint:} First show that if $F$ is a holomorphic function on $D(Q,\epsilon)$, then
\[|F(Q)|^2 \leq \frac{1}{\pi \epsilon^2} \int_{D(Q,\epsilon)}|F(z)|^2 dxdy\]
by using the Cauchy integral formula for $F^2$ and writing the integral in polar coordinates centered at $Q$; cf. Exercise 33 in Chapter 3. In detail, observe that
\begin{align*} 
\int_{D(Q,\epsilon)}|F(z)|^2 dxdy &= \int_0^\epsilon \left( \int_0^{2 \pi} |F^2(re^{i\theta})|d\theta \right) r dr\\
&\geq \int_0^\epsilon \left| \int_0^{2 \pi} F^2(re^{i \theta}) d\theta\right| r dr\\
&= \int_0^\epsilon |2 \pi F^2(Q)|rdr\\
&= \pi \epsilon^2|F^2(Q)|.
\end{align*}
Use this to show that $F$ as in the statement of the problem has at most a simple pole or removable singularlity at $P$ (not an essential singularity). Then show that $\int|F|^2 = + \infty $ if $F$ has  a pole.]
\subsection{Answer}
\begin{enumerate}
\item Let $f = \sum_{- \infty}^ \infty a_j x^j$ be the Laurent series of $f$. The function $z-pf(z)$ has series with coefficients $b_i = a_{i-1}$. Finally, by our classification of laurent series, together with the fact that the limit at $P$ is 0, we have that $b_i =0$ for $i \leq 0 \Rightarrow a_i = 0$ for $i<0$, and we have the desired result.
\end{enumerate}

\section{Problem 9}
\subsection{Question}
Prove that if $f: D(P,r) \setminus \{P\} \to \mathbb{C}$ has an essential singularity at $P$, then for each positive integer $N$ there is a sequence $\{z_n\} \subseteq D(P,r) \setminus \{P\}$ with $\lim_{n \to \infty} z_n =P$ and
\[|(z_n-P)^N \cdot f(z_n)| \geq N.\]
[Informally, we can say that $f$ ``blows up" faster than any positive power of $1/(z-P)$ along some sequence converging to $P$.]
\subsection{Answer}
\begin{proof}
Suppose not. Then, there exists $N$ such that  for all sequences $\{z_n\}$ converging to $P$, 
\[|(z_n-P)^N \cdot f(z_n)| < N.\]

So, consider the function $g = f(z) (z_n - P)^N$. This is bounded on our region by assumption, and therefore has a removable singularity. However, this means that the Laurent series has no nonzero negative power terms, and thus $f(z) = g(z) (z-P)^{-k}$ has all but finitely many negative power terms zero, a contradiction, since this would mean that $f$ has a pole.
\end{proof}

\section{Problem 13}
\subsection{Question}
Calculate the annulus of convergence (including any boundary points) for each of the following Laurent series:
\begin{enumerate}
\item \[\sum_{j=-\infty}^ \infty  2^{-j} z^j\]
\item \[\sum_{j=0}^ \infty  4^{-j}z^j + \sum_{j = -\infty}^{-1} 3^j z^j  \]
\item \[\sum_{j=-\infty}^ \infty z^j  / j^2 \]
\item \[\sum_{j=-\infty, j \neq 0}^ \infty  z^j / j^j \]
\item \[\sum_{j=-\infty}^{10} z^j / |j|! \quad (0! =1)  \]
\item \[\sum_{j=-20}^ \infty j^2 z^j \]
\end{enumerate}
\subsection{Answer}
\begin{enumerate}
\item \[\sum_{j=-\infty}^ \infty  2^{-j} z^j = \sum_{j=-\infty}^ \infty   (\frac{z}{2})^j  \]
Converges for $0\leq |z| <2$: that is, on $D(0,2)$.
\item The series \[\sum_{j=0}^ \infty  4^{-j}z^j \] converges on $0 \leq |z|< 4$, while the series \[ \sum_{j = -\infty}^{-1} 3^j z^j  \] converges for $|z| >1/3$. Thus, taking these together, we have convergence on $A(0,1/3,4)$, but nowhere on the boundary.
\item We use the root test
\[\limsup_{j \to \infty} j^{-2 / j} = 1\]
so, the positive part of the series converges on $D(0,1)$. However, the negative part converges on $A(0,1,\infty)$, so if there is any convergence it must be on $|z|=1$. There is convergence for all such, as 
\[\sum_{j=-\infty}^\infty 1 / j^2 = 2 \sum_1^\infty 1 / j^2  \]
Hence the answer is just $A(0,1,1)$.
\item  If we split the sum, considering it as \[\sum_{j=-\infty}^ 1 z^j  / j^J + \sum_{j=1}^ \infty z^j  / j^J \] it is clearly divergent, as the first piece converges, whereas the second does not.
\item We can discard the (finitely many) terms with $j >0$. Then, find the radius of convergence in $1/z$ by the root test
\[1 / \limsup_{j \to \infty } \sqrt[j] {|j|!}  = 0\]
Thus, it converges nowhere.
\item We just discard negative terms and apply the root test to get
\[ 1/ \limsup_{n \to \infty} \sqrt[j]{j^2}  = 1\]
So, we have convergence on $D(0,1)$. Clearly there is no convergence for $|z| =1$.
\end{enumerate}

\section{Problem 15}
\subsection{Question}
Make the discussion of pp. 117-118 completely explicit by doing the following proofs and continuing with Exercise 16:
\begin{enumerate}
\item Prove that if $f$ is holomorphic on $D(P,r) \setminus \{P\}$ and $f$ has a pole at $P$, then $1/f$ has  a removable singularity at $P$ with the ``filled in" value of $1/f$ at $P$ equal to $0$.
\item Let $k$ be the order of the zero of $1/f$ at $P$. Prove that $(z-P)^kf$ has a removable singularity at $P$.
\item Conversely show that if $g$ is holomorphic on $D(P,r) \setminus \{P\}$, if $g$ is \emph{not} bounded, and if there is a $m < 0 \in \mathbb{Z}$ such that $(z-P)^m g$ is bounded, then $g$ has a pole at $P$. Prove that the least such $m$ is precisely the order of the pole.
\end{enumerate}
\subsection{Answer}
\begin{enumerate}
\item 
The function $1/f$ is holomorphic on some disk around $P$, except potentially at $P$, since $f$ is.

Since $f$ has a removable singularity at $P$, for any $n > 0$ there exists some $\epsilon >0$ such that $|f(x)| > n$ for all $|x-P| < \epsilon$. Hence, taking $1/f$ we get convergence to 0. Also, observe that $1/f$ is bounded in modulus (by $1/n$) on this disk of radius $\epsilon$. Thus, by Riemann's Theorem, there is a removable singularity, and the ``filled in" value is 0 since this is the limit we computed.
\item By definition, $f$ has a pole of order $k$. Then $f (z-P)^k$ has a Laurent series expansion with coefficients $a_j = 0$ for all $j <0$, and thus has a removable singularity as claimed.
\item If the assumption is satisfied then $(z-P)^m g$ has a removable singularity by the Riemann removable singularity theorem. Hence, it has a Laurent series expansion with $a_j = 0$ for all $j<0$. Dividing by $(z-P)^m$ we have a Laurent series for $g$ which has $a_j= 0$ for all $-\infty < j < -k$. This is the Laurent series of a function with a pole of order $k$ or less.

 If we use the minimal $k$, we get precisely the order.
\end{enumerate}

\section{Problem 18}
\subsection{Question}
Let $f: \mathbb{C} \to \mathbb{C}$ be a nonconstant entire function. Define $g(z) =f(1/z)$. Prove that $f$ is a polynomial if and only if $g$ has a pole at 0. In other words, $f$ is transcendental (nonpolynomial) if and only if $g$ has an essential singularity at 0.
\subsection{Answer}
Clearly, if $f$ is a polynomial, $g$ has a pole at zero. In particular,
\begin{proof}
If $f$ is a polynomial, its modulus diverges at infinity. Since, $f(\epsilon) = f(1/\epsilon)$, and $1/\epsilon \to \infty$ as $\epsilon \to 0$, this suffices to show that $g$ has a pole at 0.
\end{proof}

Conversely, if $g$ has a pole at 0, then $f$ is a polynomial.
\begin{proof}
Note that since $f$ is entire and holomorphic, it has a Taylor series which converges everywhere. Moreover, our definition of $g$ means that its Laurent series has coefficients $a_j = b_{-j}$ for $b_{j}$ the Taylor series coefficients of $f$. 

Since $g$ has a pole, it has $a_j = 0$ for all $- \infty < j < -k$. Moreover, as it arises from a Taylor series as explained above, it has $a_j=0$ for $j> 0$. Thus, only finitely many coefficients for the Taylor series of $f$ are nonzero, and $f$ is a polynomial.
\end{proof}

\section{Problem 34a}
\subsection{Question}
Use the calculus of residues to compute the following integral:
\[\frac{1}{2 \pi i} \oint_{\partial D(0,5)} f(z) dz\]
where 
\[f(z) = \frac{z}{(z+1)(z+ 2 i )}.\]
\subsection{Answer}
The function $f$ is holomorphic on $D(0,5) \setminus \{-1, -2i\}$. Thus, the integral is by the residue theorem
\begin{align*}\mbox{Res}(-1) \cdot \mbox{Ind}(-1) + \mbox{Res}(-2i) \cdot \mbox{Ind}(-2i) &= \mbox{Res}(-1) + \mbox{Res}(-2i) \\
&=\frac{-1}{-1+2 i}+ \frac{-2i}{-2i+1} \\
&= 1\end{align*}
where the second equality comes from Prop 4.5.6.

\section{Problem 34d}
\subsection{Question}
Use the calculus of residues to compute the following integral:
\[\frac{1}{2 \pi i} \oint_\gamma f(z) dz \]
where 
\[f(z) = \frac{e^z}{z(z+1)(z+2)}\]
and $\gamma$ is the negatively (clockwise) oriented triangle with vertices $1\pm i$ and $-3$.
\subsection{Answer}
Our $f$ is holomorphic on  $\{$interior of $\gamma\} \setminus \{0,-1,-2\}$. Thus, by the residue theorem the integral is just 
\begin{align*}\mbox{Res}(0) \cdot \mbox{Ind}(0) + \mbox{Res}(-1) \cdot \mbox{Ind}(-1) \\
+ \mbox{Res}(2) \cdot \mbox{Ind}(2) &= \mbox{Res}(0) + \mbox{Res}(-1) + \mbox{Res}(-2)  \\
&=\frac{1}{2}-\frac{1}{e}+ \frac{1}{2e^2}\\
&= \frac{e^2}{2e^2} - \frac{2e}{2 e ^2} +\frac{1}{2 e^2}\\
\end{align*}
Where again the second equality comes from Prop 4.5.6.

\section{Problem 34g}
\subsection{Question}
Use the calculus of residues to compute the following integral:
\[\frac{1}{2 \pi i} \oint_\gamma f(z) dz\]
where
\[f(z) = \frac{\sin z}{z(z+2i)^3}\]
and $\gamma$ is as in Figure 4.12.
\subsection{Answer}
Our $f$ is holomorphic on  $\{$interior of $\gamma\} \setminus \{0\}$. Thus, by the residue theorem the integral is just 
\begin{align*}\mbox{Res}(0) \cdot \mbox{Ind}(0) &=  \frac{\sin 0}{(2i)^3}\\&=0
\end{align*}
Where again the second equality comes from Prop 4.5.6.



\section{Problem 47}
\subsection{Question}
Use the calculus of residues to compute the integral
\[\int_{-\infty}^{+\infty} \frac{\cos x}{1+x^4} dx.\]
\subsection{Answer}
Let's choose $\gamma_R^1$ and $\gamma_R^2$ as in Figure 4.4. That is, $\gamma_R^1$ is a straight line from $-R$ to $R$, and $\gamma_R^2$ is the upper half of the circle centered at the origin and passing through these two points.

Integrating 
\[\oint_{\gamma_R^1 + \gamma_R^2} \frac{\cos x}{1 + z^4}\]
we have by the residue theorem
\begin{align*}\mbox{Res}(e^{\pi i / 4  }) \cdot \mbox{Ind}( e^{\pi i / 4  }) +  \mbox{Res}(e^{ 3 \pi i / 4  }) \cdot \mbox{Ind}( e^{3 \pi i / 4  }) &=  \mbox{Res}(e^{\pi i / 4  }) +  \mbox{Res}(e^{ 3 \pi i / 4  }) \\
& = \frac{\cos e^{\pi i / 4  } }{1 \cdot i \cdot (1 + i)}+ \frac{\cos e^{3 \pi i / 4  } }{-1 \cdot i \cdot (-1 + i)}\\
& = \frac{\cos e^{\pi i / 4  } }{  i -1}+ \frac{\cos e^{3 \pi i / 4  } }{i +1}\\\end{align*}
The integral over $\gamma_R^2$ is bounded above by $ (\pi R)  / (1 + R^4)$, which goes to 0 in the limit. Therefore, the value computed above is the integral of just the portion we want.

\section{Problem 52}
\subsection{Question}
Use the calculus of residues to compute the integral
\[\int_0^\infty \frac{1}{p(x)} dx\]
 where $p(x)$ is any polynomial with no zeros on the nonnegative real axis.
\subsection{Answer}
We just use contours that go from $0$ to $R$ along the real axis, and from $0$ to $R$ along a circle in the 1st quadrant. 

The total integral is given by the residue theorem to be 
\[2 \pi i\]
Clearly the contour away from the real line has integral zero, hence this is the answer we want.

\end{document}
