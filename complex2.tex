\documentclass[11pt]{article}




\usepackage{fancyhdr}
\usepackage{amsthm}
\usepackage{amssymb}
\usepackage{amsmath}
\usepackage{setspace}
\pagestyle{fancyplain}



\newtheorem{theorem}{Theorem}[section]
\newtheorem*{theorem*}{Theorem}
\newtheorem{lemma}[theorem]{Lemma}
\newtheorem{proposition}[theorem]{Proposition}
\newtheorem{corollary}[theorem]{Corollary}
\newtheorem{definition}{Definition}
\begin{document}

\lhead{Frederick Robinson}
\rhead{Math 410-3: Complex Analysis}



\title{Homework 2}
\author{Frederick Robinson}
\date{11 April 2011}
\maketitle

\section*{Chapter 2}


\section{Problem  3}
\subsection{Question}
Let $U \subseteq \mathbb{C} $ be an open disc with center 0. Let $f$ be holomorphic on $U$. If $z \in U$, then define $\gamma_z$ to be the path
\[\gamma_z(t) = tz, \quad 0 \leq t \leq 1.\]
Define
\[F(z) = \oint_{\gamma_z} f(\zeta) d\zeta. \]
Prove that $F$ is a holomorphic antiderivative for $f$.
\subsection{Answer}
\begin{proof}
We proved in Theorem 2.3.2 that the function $F'$ acquired by integrating
\[F' = \int_0^{\Re(z)} \Re{f} + \int_0^{\Im(z)}  \Im(f) \]
is has the desired properties. It therefore suffices to show that $F' = F$. This is just a straightforward application of the Cauchy Integral Theorem though. 

Observe that
\[F' = \int_{\gamma'} f\]
where $\gamma'$ is the union of the line segments from 0 to $\Re z$ and $\Re z$ to $\Im z$.  If we denote by $\gamma' - \gamma$ the path acquired by traveling along $\gamma'$, then backwards along $\gamma$, the Cauchy Integral Theorem gives us
\[0 = \oint_{\gamma' - \gamma} f(\zeta) d \zeta = \oint_{\gamma'} f(\zeta) d \zeta - \oint_\gamma f(\zeta) d \zeta  = F' - F\]
which is what we wanted to show.
\end{proof}

\section{Problem  4a}
\subsection{Question}
Compute
\[\oint_\gamma \frac{1}{z} dz\]
where $\gamma$ is the unit circle (center 0) with counterclockwise orientation.
\subsection{Answer}
The Cauchy integral formula with $z =0, f(z) = 1$ is
\[1 = \frac{1}{2 \pi i } \oint_\gamma \frac{1}{\zeta} d \zeta.\]
Therefore the desired integral is just $2 \pi i$.

\section{Problem  4c}
\subsection{Question}
Compute
\[\oint_\gamma \frac{z}{8+z^2} dz\]
where $\gamma$ is the triangle with vertices $1,i,-i$ and $\gamma$ is equipped with counterclockwise orientation.
\subsection{Answer}
The zeroes of the denominator lie outside the triangle. The function is therefore holomorphic on the interior, and the integral evaluates to
\[\oint_\gamma \frac{z}{8+z^2} dz = 0.\]


\section{Problem  5}
\subsection{Question}
Evaluate 
\[\oint_\gamma z^j dz,\]
for every integer value of $j$, where $\gamma$ is a circle with counterclockwise orientation and whose interior contains 0.
\subsection{Answer}
In the case of $j=-1$ we solved this in exercise 4a. Assuming $j \neq -1$, the function is holomorphic on $\mathbb{C} \setminus \{0\}$. Hence,
\[\oint_\gamma z^j dz = \left\{ \begin{array}{ll} 2 \pi i & j=-1\\ 0 &\mbox{otherwise} \end{array}\right..\]


\section{Problem  18b}
\subsection{Question}
Compute
\[\oint_\gamma \frac{\zeta}{(\zeta+4)(\zeta-1+i)} d \zeta\]
where $\gamma$ describes the circle of radius 1 with center 0 and counterclockwise orientation.
\subsection{Answer}
As the function is  holomorphic on the interior of the curve we just have
\[\oint_\gamma \frac{\zeta}{(\zeta+4)(\zeta-1+i)} d \zeta = 0\]

\section{Problem  18d}
\subsection{Question}
Compute 
\[\oint_\gamma \zeta(\zeta+4) d\zeta\]
where $\gamma$ is the circle of radius 2 and center 0 with clockwise orientation.
\subsection{Answer}
As $\zeta^2 + 4 \zeta$ is a polynomial, it's holomorphic. Hence, by the Cauchy Integral Theorem
\[\oint_\gamma \zeta(\zeta+4) d\zeta = 0.\]


\section{Problem  18e}
\subsection{Question}
Compute
\[\oint_\gamma \overline{\zeta} d\zeta\]
where $\gamma$ is the circle of radius 1 and center 0 with counterclockwise orientation.
\subsection{Answer}
We can compute the integral as
\begin{align*}\oint_\gamma \overline{\zeta} d\zeta &= \oint_\gamma \Re \overline{\zeta} d\zeta + i \oint_\gamma \Im \overline{\zeta} d\zeta\\
&=  \oint_\gamma \Re \zeta d\zeta - i \oint_\gamma \Im \zeta d\zeta.
\end{align*}
However since the function $f(z) = z $ is holomorphic, we have $0-0=0$.
\section{Problem  28a}
\subsection{Question}
Compute explicitly the integrals
\[\oint_{\partial D(8i,2)} z^3 dz,\]
\[\oint_{\partial D(6+i,3)} (\overline{z} - i ) ^2 dz.\]
\subsection{Answer}
\begin{align*}
\oint_{\partial D(8i,2)} z^3 dz &= \int_0^1 (8 i + 2 e^{2 \pi i t })^3 2e^{2 \pi i (t+ 1/4 )} dt\\
&=  \int_0^1 -512 i e^{2 i \pi  t}-384 e^{4 i \pi  t}+96 i e^{6 i \pi  t}+8 e^{8 i \pi  t} dt\\
&=0
\end{align*}
For the second computation we have
\begin{align*}
\oint_{\partial D(6+i,3)} (\overline{z} - i ) ^2 dz &= \int_0^1  ( (6 - i+3e^{ - 2 \pi i t}) - i ) ^2  3e^{2 \pi i (t+ 1/4 )}  dt\\
&= \int_0^1 (36+108 i)+27 i e^{-2 i \pi  t}+(72+96 i) e^{2 i \pi  t}  dt\\
&= 36 + 108 i
\end{align*}


\section{Problem  43}
\subsection{Question}
If $f$ is a holomorphic polynomial and if
\[\oint_{\partial D(0,1)} f(z) \overline{z}^j dz =0, \quad j=0,1,2,\dots,\]
then prove that $f \equiv 0$.
\subsection{Answer}





\end{document}
